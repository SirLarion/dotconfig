\section{Introduction}
% --------------------------------------------------------------------

Blockchain has become a point of interest for people of all walks of
life. While the underlying technology was conceived by Haber and
Stornetta already in 1990 \cite{haber1990time}, public interest in it
mostly arose around 2016 and 2017. This was when cryptocurrencies,
digital adaptations of cash, which are often implemented with
blockchains, saw large popularity growth as financial instruments. Due
to the privacy-protecting nature of cryptocurrencies, it is difficult
to estimate how many individual owners of cryptocurrencies there are.
One report \cite{bbtcStats} states, though quite baselessly, that in
2021, over 300 million people use or own cryptocurrencies globally. A
study by Blandin et. al. in 2020 \cite{blandinBenchmark} found that
globally around 101 million people owned cryptocurrencies. While
interest in cryptocurrencies also indirectly means interest in the
blockchain technology, this development has lead to a situation where
many people think blockchain and cryptocurrencies are one and the same
[SOURCE]. However, the envisioned applications for blockchains range
far wider than finance. A 2018 paper \cite{kasparsUseCases} collected
a set of different areas where blockchain was in the process of being
applied. These include topics from identity management to insurance to
voting and video games.

The diversity of the proposed applications is exciting. However, this
highlights how susceptible blockchain is to the \textit{"silver bullet
syndrome"} of new technologies. Many proponents of blockchain see it
as the solution to solve all problems. Consequently, there are
countless blockchain startups and projects popping up and attempting
to attract investors with big, yet baseless, promises. This
makes it difficult to find those projects which are to be taken
seriously and which can show in practice what kind of value there is
to be derived from the new technology. Another consideration to keep
in mind with, really, any new technology is whether the problems it
solves can already be solved with existing technology. This is
important because new technologies are often adopted simply for their
novelty even though existing, time-tested, technologies could do the
job better. A vocal critic of blockchain, Jorge Stolfi, has stated
that blockchain is a solution looking for a problem
\cite{stolfiNISTReview}.

This thesis aims to find the strengths and limitations that arise from
the technical implementation of blockchain. It explores the
\textit{proposed} applications of blockchain but also its
\textit{realized} applications, which have been found to produce value
to its users in practice.

% --------------------------------------------------------------------