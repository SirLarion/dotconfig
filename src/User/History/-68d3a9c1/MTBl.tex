\section{Applications of blockchains}

Given the strengths and weaknesses of blockchains, what can they be
used for in practice? I will spend this chapter discussing what are
the potential use cases of blockchains. I will go through a set of
applications that are used in different fields of work. I will discuss
only the applications that are already established. Speculation about
the potential value of applications in development is not in the scope
of this study. 

The section is structured such that each subsection concerns a domain
where blockchain has been taken into use. Each domain contains several
blockchain applications that are discussed. For each application (or
several applications if the comparison is feasible), a counterexample
that does \textit{not} utilize blockchain is presented.  
Finally, at the end of each subsection, I will review the
applicability of blockchain in the domain from the perspective of both
the studies that have been written on it and its relative performance
when compared to the given counterexamples. This is done to avoid
force-fitting blockchain solutions to problems that could be solved
better in other ways. 

I will also broadly review the maturity of the literature concerning
each domain.

...

\subsection{Payment systems}

In the context of blockchain applications, the domain of payment
systems relates closely to financial instruments. However, I make a
clear distinction here between the two. This is primarily because the
domain of cryptocurrencies has become proliferated with different
'coins' that, though theoretically usable for payment, are developed
with the sole purpose of being used as speculative financial
instruments. 

\paragraph{Example: Bitcoin}

Satoshi Nakamoto proposed Bitcoin as a way 

\paragraph{Counterexample: Visa}

\subsection{Intrabank payments}

Also, intrabank payments is another problem space where
blockchain has often been proposed as a solution. Neither of these
areas benefits from a focus on creating financial instruments.


\paragraph{Example: ...}

\subsection{Digital supply chain management}

In markets where companies increasingly outsource parts of their
production to different, often international, entities, managing the
resulting supply chains (SC) is instrumental to all entities taking
part. A high degree of planning and communication is required to
respond to future demand for products and maintaining
cost-effectiveness throughout the chain.

As such, supply chain management (SCM) is an area which would
certainly benefit from digitalization. Korpela et. al.
\cite{korpelaDSC} discuss the potential benefits of digital supply
chains (DSC). These include improving SC effectiveness, reducing costs
throughout the chain and providing flexibility in designing them. Done
properly, digitalization would also decrease human error as a
consequence of automation. However, converting to DSCs is quite a
difficult problem. A fully digitalized SC would require
interoperability between the internal systems of \textit{all} the
participants in the SC.

Korpela et. al. go on to explore potential benefits of using
blockchains to drive business-to-business (B2B) system integration
within DSCs. They argue that in DSCs, creating a blockchain layer
would allow for interoperability between businesses while providing
visibility to the current state of the chain.

A permissioned blockchain enables a DSC to operate using trusted
sensors whose data are added to the ledger for SC participants to see.
They can then be confident that the data has not been tampered with.
However, what if the \textit{environment} of the sensor is tampered
with? \cite{wustBlockchainNeed} gives an example of a blockchain-based
DSC for food. In one part of it, food is meant to be stored in a
certain temperature during delivery. Figure X illustrates the posed
problem. The sensor that the blockchain trusts to give an accurate
reading of the temperature the food is stored in, is itself located in
a cooled section while the food is not. This way, a dishonest
participant of the SC can cut costs by having reduced cooling for the
food while the blockchain indicates it was properly cooled. This
problem can, of course, be fixed by verifying through some third party
that the sensors are installed correctly. But if such a system of
verification is available, why not simply use a regular database
management system for reading and writing the data from the sensors?


A case study by the Hyperledger Foundation \cite{hyperledgerHitachi}
investigates how Hitachi used the Hyperledger Fabric blockchain framework
to improve their internal contract processing procedure. They started
from a completely paper-based system and converted a division to use
a fully electric signature system using Hyperledger Fabric. As a result,
Hitachi reported a 20\% increase in the amount of processed contracts.



\subsection{Internet of Things}

\subsection{Identity management}

\subsection{}