\section{Background}
% --------------------------------------------------------------------
First, it is important to know why the technology for blockchains was
created. Has the reasoning changed over time? Establishing this base
is instrumental. It will help with evaluating, first, the necessity of
using blockchain for different applications and, second, how well
existing blockchain applications solve the problems they were created
for.

\subsection{Motivation}
Back in 1990, in one of blockchain's foundational papers "How to
Time-stamp a Digital Document" \cite{haber1990time}, Haber and
Stornetta proposed a solution for the problem of verifying when and by
whom a publicly accessible ledger was last modified. This issue of
verifying the integrity and authorship of a file are fundamental
issues in cybersecurity. Today they are addressed quite universally by
creating signatures via standardized hashing algorithms such as that
of OpenPGP \cite{PGPstandard}. 

Can the signature, or the file itself, be trusted? Either of them
might have been created maliciously or in an otherwise faulty way
simply by accident. For example, in the context of finance, this
results in entities being able to create nonexistent transactions with
the impression of legitimacy. This is known as the double spending
problem. Typically, this problem is solved by consulting a trusted
third party, such as a bank, that works as an intermediary to make
sure that transactions are conducted as agreed upon. Of course, this
requires that a trusted third party is \textit{always} available,
meaning that they have to be consulted regarding each transaction that
is made. Quite often, this is not the case. In a \textit{limited
trust} environment, the involved parties have to reach a consensus by
themselves about which transactions are legitimate.

This is where blockchain comes in. The consensus protocol at the core
of each blockchain implementation enables participants in a limited trust
environment to, nevertheless, have trust in the transactions that are
recorded.

Haber and Stornetta, among several others \cite{haber1990time,
parliamentLamport, benalohBroadcast}, had built a strong mathematical
base for the blockchain technology. The first proper use case of it
was only missing public interest in solving the problem of
\textit{"trustless trust"}. Later, in 2008, the market crashed,
leading to strong distrust in centralized financial authorities. As a
result, pseudonym Satoshi Nakamoto proposed their electronic
equivalent for cash, Bitcoin \cite{nakamoto2008bitcoin}.

...


\subsection{The technology}

Blockchain is a technology with many definitions. It is, increasingly,
a category of technologies rather than a singular one. Generally, all 
blockchains have a few key concepts at their core:
\begin{enumerate}
  \item Recordkeeping with a ledger
  \item Network permissions
  \item Linked blocks as data containers
  \item Protocol for reaching concensus
\end{enumerate}
Each blockchain has its own quirks regarding these concepts. Next, I
will expand upon each item in the list. Finally, I will discuss some
aspects of blockchain that are left out of the list above, but are
still worth mentioning in this context. 

During this section, for the sake of simplicity, I will explain
concepts in the blockchain technology mainly through the use case of
monetary transactions. This is because I feel it is the most easily
understandable practical example. Note, however, that all these
concepts can be generalized. Later, in section 4, I will show
how they can be expanded to different appications.

\subsubsection{Recordkeeping with a ledger}
All blockchains are digital, append-only, ledgers. This means that
transactions are stored as a sequential list of records. Each record is
immutable and can only be undone by appending the "reverse" of that
record to the ledger. For example, if Alice sends 100€ to Bob, this
transaction can only be reverted by creating another transaction where
Bob sends the same amount back to Alice.

...

\subsubsection{Network permissions}
When creating a blockchain, it must be defined who has the permission
to access it. As such, blockchains can be divided to two categories:
permissionless and permissioned blockchains.

A permissionless blockchain can be accessed by anoyone. Most
cryptocurrencies, for example, are permissionless, as enabling anyone
to create transactions with them is a core principle for them. In the
network of a permissionless blockchain, all participants are distrusted.
This leads to higher requirements for the concensus mechanism as
verifying the legitimacy of transactions becomes more difficult.

Permissioned blockchains can only be read and written to by entities
who have been given permission to it. In practice, a permissioned
blockchain could be a ledger of internal transactions inside a
corporation. Clearly, in a permissioned blockchain, the requirement
for trust is weaker as it can be assumed that a notable portion of
the participants can be trusted.

...

\subsubsection{Linked blocks as data containers}
As the name implies, a blockchain is composed of "blocks" that are
linked together. In the beginning there is just one block which simply
contains any data, say, details for a set of transactions. Any
following transactions can be added to the blockchain by taking the
hash value of the previous block and including it alongside the new
set of transactions. This then becomes the new block which can again
be hashed and included in the next one.

Each block also contains some crucial metadata that give the
technology its resistance to tampering. Blocks have an identifier to
differentiate them from other blocks. A timestamp is also included. This,
in conjunction with the ID, hash function and concensus model is
what verifies that the blockchain has not been modified.

...


\subsubsection{Protocol for reaching concensus}
Blockchain networks do not have a trusted central authority to
validate which blocks should be included in the chain. Due to this, a
concensus must be reached between the participants of the network on
which blocks to include. As participants are assumed to be distrusted,
reaching a concensus becomes quite difficult. Many different models
have been proposed to act as the mechanism by which the inclusion of
blocks is decided.

Proof of work (PoW) blockchains require the participants to solve a very
specific "puzzle". Whoever solves this puzzle gets to decide which
block is added to the chain. Bitcoin, for example, requires that, to
add a block, a participant of the network needs to produce a hash
value that starts with a certain number of '0' bits
\cite{nakamoto2008bitcoin}. 

In proof of stake (PoS) blockchains, participants must pay to take
part in the decision making process. Most often, this roughly
translates to participants buying a certain amount of "voting" power.
This amount then correlates with how much the participant can affect
the decision of which block gets included next.

...

% --------------------------------------------------------------------