% Tiivistelmät tehdään viimeiseksi. 
%
% Tiivistelmä kirjoitetaan käytetyllä kielellä (JOKO suomi TAI ruotsi)
% ja HALUTESSASI myös samansisältöisenä englanniksi.
%
% Avainsanojen lista pitää merkitä main.tex-tiedoston kohtaan \KEYWORDS.

\begin{fiabstract}
Lohkoketjuteknologia (\textit{engl. blockchain}) on herättänyt paljon
sekä julkista että akateemista kiinnostusta viime vuosina. Kiinnostus
on kuitenkin kohdistunut suurimmalta osin kryptovaluuttoihin eli
yhteen lohkoketjujen käyttökohteista, vaikka monta muuta mahdollista
käyttökohdetta on ehdotettu. Tästä seurauksena on syntynyt aukko
julkisessa tietämyksessä sekä akateemisen tutkimuksen kohdistamisessa
lohkoketjujen ja kryptovaluutoiden välillä.

Tämä kandidaatintyö tutkii, mitä erilaisia lohkoketjuja on olemassa ja
mitkä ovat niiden vahvuudet ja heikkoudet. Työ myös käsittelee useaa
ehdotettua käyttökohdetta lohkoketjuille ja arvioi löydettyjen
vahvuuksien ja heikkouksien perusteella, onko lohkoketju toimiva
ratkaisu kyseiseen tarkoitukseen vai kannattaisiko käyttää sen sijaan
jotakin muuta teknologiaa.

Työssä ehdotetaan, että lohkoketjuille on toimivia käyttökohteita
esimerkiksi tuotantoketjujen sekä maksujärjestelmien alueilla, mutta
käyttöä on lähestyttävä tarkkaavaisesti. Työn mukaan lohkoketjujen
toimivuus riippuu suuresti käyttökohteen vaatimista ominaisuuksista.
Erityisesti skaalautuvuus onn merkittävä tekijä käytettävää
teknologiaa valittaessa.

%
%Tiivistelmätekstiä tähän (\languagename). Huomaa, että tiivistelmä tehdään %vasta kun koko työ on muuten kirjoitettu.
\end{fiabstract}

%\begin{svabstract}
%  Ett abstrakt hit 
%%(\languagename)
%\end{svabstract}

\begin{enabstract}
Blockchain has become a considerable point of interest both in public
and academic discussion during the last years. However, the focus of
this interest has largely been directed at crypto-assets, a use case
of blockchains, though many other use cases have been proposed.
Consequently, a gap, both in public knowledge and academic research,
has formed between crypto-assets and blockchain. 

This bachelor's thesis examines different kinds of blockchains and the
strengths and weaknesses that they come with. It additionally explores
several problem spaces where blockchain has been proposed for use.
Assessments of blockchain's applicability are made based on the found
strengths and weaknesses in order to evaluate whether blockchain or
another existing technology should be used as a solution.

The paper suggests that blockchains seem to have viable areas of
application, for example in supply chain management and in payment
systems. Yet, applying it must be approached with caution. This is
because blockchain's applicability is heavily dependent on the
properties required by each use case. Especially, scalability is
identified as a significant factor to take into count when deciding
the used technology.

%%(\languagename)
\end{enabstract}
