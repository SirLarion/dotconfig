\section{Blockchain applicability} \label{applications}

The list of proposed blockchain application domains is long
\cite{zileUseCases}. For the scope of this study, reducing it to a
significantly shorter one is necessary in order to conduct in-depth
examinations of each domain. Based on findings from \cite{jaoudeApplications} 
and assessments of several studies that examine blockchain use cases
\cite{sankaScalability,wustBlockchainNeed,zhengChallenges}, the
domains discussed are:

\begin{enumerate}
  \item Digital payment systems
  \item The Internet of things
  \item Supply chain management
\end{enumerate}

I will discuss challenges associated with each domain. Then I will present
ways that using blockchain may benefit the domain with regard to the
challenges and otherwise. After this, I will present alternative ways
to approach these challenges and conclude on an assessment of blockchain's
applicability with a specification on what kind of blockchain seems most
suitable.

\subsection{Digital payment systems}

In a contrast to history, modern digital payment systems are
incredibly efficient and secure. Verifying the legitimacy of the money
that exchanges hands in a payment scenario is practically trivial. As
these exchanges happens digitally and in a fully encrypted manner, any
consideration for whether someone steals the money en-route to its
recipient is unnecessary. Currently, this is enabled by the operation
of two entities: the bank, who is housing the money that the buyer has
and the seller will have, and the \textit{payment service provider}
(PSP), who maintains the system which asks the bank to move money from
one account to another. This system is thus dependent on a two-pronged
assumption:
\begin{enumerate}
  \item The bank is trusted and always available
  \item The PSP is trusted and always available
\end{enumerate}

\paragraph{Challenges} Two  issues arise from this. Payments
across national borders is complicated when different nations utilize
different currencies. A more critical issue is that fiat currencies
are monopolized to a single issuer \cite{eichengreenCommodity}. As
money is an all-encompassing element of modern life, this makes quite
an existential assumption on the trustworthiness of the issuer.. 

\paragraph{Blockchain solutions} The initial motivation for
creating crypto-assets was to use them as means of payment as proposed
by Bitcoin \cite{nakamoto2008bitcoin}, this idea has not seen much
interest since its conception \cite{liptonTether}. In fact, several
authors have argued that crypto-assets do not meet the criteria to be
considered currencies at all due to their strong fluctuations in price
\cite{carstensMoney,merschVirtual,yermackBitcoin}. 

Generally, use of crypto-assets has been shown to exist largely in the
context of speculative finance \cite{liptonTether,steinmetzOwnership}.
However, stablecoins, a fairly recent development in the
crypto-asset space, have a focus on price stability
\cite{mitaStablecoin} as the name implies. Due to this, they are
considered to hold the potential to be used as digital currencies
\cite{bullmannSearch,liptonTether}. Lipton et al. go so far as to
claim that stablecoins are conceptually much larger than crypto-assets
as a potential generic form of digital money.

Mita et al. \cite{mitaStablecoin} review existing stablecoins and
their ways of attaining stability. The system for stabilizing a coin's
value can be based on providing other assets -- US dollars, other
crypto-assets or gold, for example -- as collateral. Stabilization
can also be \textit{non-collaterized}. Here, the stabilization
mechanism is purely an algorithm which automatically attempts to
control the amount of stablecoins in circulation to stabilize the
price. The terms \textit{non-collaterized} and \textit{algorithmic}
are often used synonymously in the literature. It must be noted that
not all algorithmic stablecoins are non-collaterized as they can use
both the algorithm and some form of collateral for stabilization.

Mita et al. go on to argue that only non-collaterized stabilization
can have the desired outcome for stablecoins. They say collaterization
with national-backed currencies or commodities would result in
centralization and collaterization with other crypto-assets would,
in turn, not actually stabilize the value of the currency. Several
other authors, however, are sceptical of non-collaterized stablecoins
\cite{berentsenStablecoins,bullmannSearch,eichengreenCommodity,lyonsStable}.
Eichengreen \cite{eichengreenCommodity}, for example, takes a firm
stance against non-collaterization. Eichengreen argues that
non-collaterized stablecoins cannot have resilience against price
fluctuations as they lead to snowball effects that, without
collateral, easily collapse the value of the coin entirely. Berentsen
and Schär \cite{berentsenStablecoins}, on the other hand, consider
stablecoins collaterized with crypto-assets
(DAI\footnote{https://makerdao.com/en/}, for example) to be the
perfect middle-ground between decentralization and stability.

The blockchain-based stablecoins that are currently traded most are
collaterized. At the time of writing, data from CoinMarketCap
\footnote{https://coinmarketcap.com} shows that the ten stablecoins
with the highest trading volume include three stablecoins that are
listed as "algorithmic" (Decentralized USD, Terra Classic USD and
sUSD). These account for about 0.6\% of the trading volume of the top
10 while the top three (USD Tether, USD Coin and Binance USD) account
for roughly 98\% of the same volume.

Tether\footnote{https://tether.to}, specifically USD Tether (USDT),
is by far the most traded stablecoin, accounting for 77\% of the
volume of the top ten stablecoins by volume in CoinMarketCap
statistics. In fact its trading volume is higher than that of Bitcoin
(BTC), the second most traded crypto-asset, by about 74\%. It is
collaterized by national-backed currencies which are held by Tether
Holdings Limited, a daughter company of Tether Limited. Tether Limited
is the issuer for so called "Tether tokens". When paying an amount of
a certain currency to Tether Limited, they issue a corresponding
amount of Tether tokens pegged to that currency. For example, USDT are
Tether tokens backed by US dollars and EURT respectively by
Euros. These Tether tokens exist on multiple different crypto-asset
blockchains to create price stability. Since the issuance of USDT in
2015, its deviations from the price of USD have remained under ±10\%
Figure \ref{fig:tether-deviations} from a paper by Lyons and
Viswanath-Natraj \cite{lyonsStable} displays the deviations in more
detail.

DAI, a significantly less traded stablecoin (about 0.5\% of Tether's
trading volume), is collaterized with other crypto-assets. DAI is
pegged to US dollars like Tether.

% ...

% More examples \& counterexamples

% ...

\begin{figure}
  \includegraphics[width=\linewidth]{tether-deviations.png}
  \caption{USDT/USD deviations from peg and histogram of deviations.
  As shown in \cite{lyonsStable}.}
  \label{fig:tether-deviations}
\end{figure}

\paragraph{Alternative solutions}

\paragraph{Conclusion}

\subsection{Internet of things}

The \textit{Internet of things} (IoT) often refers to a global network
of interconnected devices, \textit{IoT nodes}, with digital capabilities such
as weather sensors, home appliances and even cars. "IoT" is additionally
used to describe similar, but \textit{local} intranets that are not
connected to the global Internet. 

Advocates of \textit{consumer-centric} IoT suggest that it can enable
the development of \textit{smart cities}, smart energy grids
\cite{reynaIntegration}, intelligent transportation systems and better
monitoring of, for example, pollution and weather
\cite{sisinniIndustrial}. Moreover, it is said to create a digital
representation of the physical world, increasing the degree with which
people are able to interact with the world
\cite{reynaIntegration,viriyasitavatIot}. Furthermore,
\textit{industrial} IoT (IIoT) is said to provide efficiency and
productivity to industrial systems through a higher degree of
automatization and better monitoring and management capabilities
\cite{coleBlockchain,sisinniIndustrial}.


\paragraph{Challenges} There exist a large amount of IoT nodes
(estimated tens of billions \cite{reynaIntegration}) with
\textit{highly} varying capabilities and many different configurations
for connectivity and data sharing \cite{sisinniIndustrial}.
Consequently, IoT faces a number of challenges:

\begin{itemize}
  \item \textbf{Susceptibility to security threats.} Many IoT nodes
  operate with low bandwidth and weak security measures to achieve
  better energy efficiency. Thus, defense against attacks is
  nearly non-existent \cite{sisinniIndustrial}. Additionally, updating
  the software of IoT nodes is time consuming and expensive due to 
  their large amount \cite{jaoudeApplications}.

  \item \textbf{Lack of data interface interoperability.} IoT nodes
  have varying configurations for their data interfaces
  \cite{panInteroperability,viriyasitavatIot}. As such, integrating
  different IoT systems and gaining useful aggregations of data from
  many nodes can be complicated \cite{sisinniIndustrial}.

  \item \textbf{Complexity of maintaining confidentiality.} Due to the
  lack of interoperability, maintaining the confidentiality of data,
  especially in publicly available applications, is highly complex.

  \item \textbf{Storing large quantities of data.} The key activity in
  IoT is data collection. As all IoT nodes are continuously collecting
  data, storing it becomes increasingly difficult.

  \item \textbf{Resource management.} Having a large amount of IoT
  nodes requires constant resource management. When nodes are
  operating expectedly, they require electricity to run on. When not,
  they require maintenance in order to preserve the functionality of
  the IoT system.
\end{itemize}


\paragraph{Blockchain solutions} Using blockchain within IoT is a
topic that has gained significant academic interest
\cite{jaoudeApplications}. It is considered suitable to fix many of
the problems faced within IoT. 

An often mentioned advantage in blockchain-powered IoT is enchanced
security. A literature review by Jaoude et al.
\cite{jaoudeApplications} has identified the following benefits on
security: Using a blockchain enables verifying that a node is
operating according to up-to-date instructions. If it is
malfunctioning, its software has been modified by an adversary or it
requires a system update, the blockchain can display this in the
node's data output. Furthermore, in case of a system update, the
authors show that the update can be delivered to the node through the
blockchain itself. In addition, Reyna et al. note that securing
inter-node communications in IoT can be done by validation with
smart contracts in blockchains \cite{reynaIntegration}.

Blockchain can additionally improve trust towards IoT networks. The
tamper-resistance in blockchains may be utilized to increase the
trustworthiness of IoT nodes' data \cite{jaoudeApplications}. One
problem mentioned with IoT networks is that they are opaque, leading
to distrust towards the networks, especially in consumer-centric IoT
\cite{reynaIntegration}. Blockchain can provide visibility to the
state of the network. If privacy is not critical, any blockchain
implementation will allow for data verifiability, and in
privacy-critical applications, NIZK-enabled blockchains may be used to
maintain confidentiality while enabling verifiability (Section
\ref{security}). Notably, blockchain can provide privacy overall to
IoT nodes through pseudonymity. 

Other identified advantages of blockchains in IoT include the ability
of reducing data storage requirements by enabling the IoT nodes to
process recorded data themselves through the use of smart contracts
\cite{viriyasitavatIot}. Furthermore, smart contract utilization may
enable a higher degree of autonomous activity on the part of IoT nodes
improving, for example, \textit{machine-to-machine} decisionmaking
\cite{panInteroperability} such as automatic purchasing of energy
\cite{jaoudeApplications,viriyasitavatIot}. Blockchains, as
distributed systems, additionally limit the amount of single points of
failure in the network.

New issues do emerge when using blockchain for an IoT system. Notably,
scalability is a challenge that is difficult to circumvent due to
IoT's dependence on both a large number of nodes in the network and
high throughput to write data. From Section \ref{scalability}, a
permissionless blockchain scales well in network size but is more
lacking in TPS. A permissioned blockchain, in contrast, has more
difficulty in scaling network size but can reach high TPS quite well.

Reaching consensus between IoT nodes is another problem. Consensus
mechanisms in permissionless blockchains often require computational
capacity or storage capacity, both of which can have strict limits in
IoT nodes \cite{viriyasitavatIot}. PoS consensus is an exception since
it has low demands from the node. However, PoR consensus, though it
requires storage capacity, may be considered as it could be used to
store network data \cite{banoConsensus}, mitigating the problem of big
data in IoT. Consensus models in permissioned blockchains generally
are low in system demands, but they often instead require low latency
between nodes to maintain a high TPS, such as in PBFT
\cite{tomicPermissioned}. This may be a problem as IoT nodes'
connections to the network are often of low quality. However,
permissioned consensus models may be improved in this respect as
using, for example, \textit{gossip protocols} can decrease the demand
for low latency by requiring less visibility from a node to the
network \cite{banoConsensus}. In the future, the use of 5G networks
may further decrease the prevalence of this problem
\cite{sisinniIndustrial}.

The \textit{garbage in garbage out} (GIGO) problem has been identified
by several authors as a notable challenge when using blockchains in IoT
\cite{babichDistributed,reynaIntegration,wustBlockchainNeed}.
Blockchain relies on input from the physical world being trustworthy.
If illegitimate data is entered mistakenly or maliciously, it is
likely to remain on the blockchain due to tamper-resistance
\cite{olearyConfiguring}. Wüst and Gervais \cite{wustBlockchainNeed}
illustrate this clearly as seen in Figure \ref{fig:scm-problem}. The
GIGO problem has also been identified in practice by several companies
as shown by \cite{rogersonFood}. While no clear solution to it exists,
permissioned blockchains can alleviate it somewhat due to their higher
degree of trust between nodes.

\begin{figure}
  \includegraphics[width=\linewidth]{scm-problem.png}
  \caption{A potential problem with trusted sensors in an IoT
  blockchain as posed by \cite{wustBlockchainNeed}. On the left, a
  temperature sensor is placed correctly in a food carrier truck. On
  the right, it is placed maliciously in a cooled compartment to
  falsify the recorded temperature inside the truck.}
  \label{fig:scm-problem}
\end{figure}


\paragraph{Alternative solutions} Purely as a mechanism of data
storage, blockchain is, regardless of implementation, less efficient
than established techonologies like database management systems (DBMS)
\cite{stolfiNISTReview,wustBlockchainNeed}. Especially
\cite{chenComparative} shows how even a private blockchain is less
efficient on all counts than a DBMS like MySQL. DBMSs can also be
resistant to single points of failure as their infrastructure can be
distributed. Due to this, using DBMS in IIoT when all IoT nodes are
trusted, such as in the production system of a single company, seems
suitable.

Several possible solutions have been suggested with regard to security
in IoT. Hassija et al. mention, apart from blockchain, edge computing
and fog computing; extensions of cloud computing, and machine learning
algorithms \cite{hassijaSurvey}. Each of these have their own
strengths and weaknesses which are outside this paper's scope.
However, none of these solutions is found to be clearly superior to
the others.

\paragraph{Conclusion} Based on this discussion, blockchain's
applicability to IoT cn be assessed. The result of the assessment is
shown in Table \ref{tab:iot}. The only clearly noticable edge blockchain
has over other solutions is its decentralization. This
decentralization is gained in exchange of performance always being
weaker than the alternative options and having to solve the GIGO
problem. It can thus be said that blockchain should only be used in
IoT if trust between the blockchain's writer entities is limited. In a
full-trust environment, distributed DBMS solutions should be used.

The applicability of different blockchains varies. In consumer-centric
IoT, requiring permissions may be infeasible for the application.
Thus, a permissionless consensus mechanism with low system demands and
high TPS such as DPoS can be utilized. \textit{Roll-DPoS} is an
improvement on DPoS that is specifically designed with the low
capabilities of IoT nodes in mind. It is said to achieve 3000 TPS
\cite{sankaScalability}. Additionally, using a PoR blockchain, such as
PermaCoin \cite{millerPermacoin}, has potential conceptually but this
should be studied more. The prevalence of the GIGO problem remains. It
may be alleviated by having larger networks of sensors to confirm
other sensors' data \cite{babichDistributed}. In a permissionless
blockchain, this can have significant costs but it may be the required
trade-off for public availability.

In IIoT, the situation is somewhat different. An IIoT blockchain can
require permissions from its nodes. As such, a permissioned blockchain
seems more applicable due to its capability for high TPS and low
system demand for nodes while inherently alleviating the GIGO problem.
Additionally, a permissioned blockchain is better equipped to further
reduce the impact of GIGO since several independent entities' nodes
can be used to confirm each others' data. Hyperledger Fabric, for
example, is a suitable blockchain implementation in IIoT. A study by
Kuzlu et al. supports this conclusion \cite{kuzluPerformance}.

\begin{table}[]
\begin{tabularx}{\textwidth}{lXXX}
Solution                                                        & Hyperledger Fabric & Roll-DPoS           & Distributed DBMS \\[2ex] \hline
\begin{tabular}[c]{@{}l@{}}Trust \\ required\end{tabular}       & limited            & low                 & full             \\[3ex] \hline
\begin{tabular}[c]{@{}l@{}}Permission\\ required\end{tabular}   & yes                & no                  & yes              \\[3ex] \hline
\begin{tabular}[c]{@{}l@{}}Implementing\\ security\end{tabular} & cheap              & cheap               & expensive        \\[3ex] \hline
\begin{tabular}[c]{@{}l@{}}Scaling:\\ Network size\end{tabular} & moderate           & high                & low              \\[3ex] \hline
\begin{tabular}[c]{@{}l@{}}Scaling:\\ TPS\end{tabular}          & high               & moderate            & optimal          \\[3ex] \hline
\begin{tabular}[c]{@{}l@{}}GIGO\\ problem\end{tabular}          & solvable           & difficult           & inexistent       \\[3ex] \hline
\begin{tabular}[c]{@{}l@{}}Suggested\\ area\end{tabular}        & IIoT               & Consumer IoT        & Private IIoT     \\[3ex] \hline
\end{tabularx}
\caption{Results of assessing blockchain applicability in IoT}
\label{tab:iot}
\end{table}


\subsection{Supply chain management}


Supply chain management (SCM) is the activity of planning and managing
the process of procuring raw materials and turning them into products
which are sold to customers. The process involves several different,
often independent, companies forming a supply chain (SC). SCM is
concerned with the cost-effectiveness of the chain and being able to
respond quickly to changes in the supply and demand of products. 

Digital supply chains (DSC), as opposed to traditional SCs, have been
proposed. IoT networks are key components in DSCs \cite{panInteroperability}. As such, many of
their advantages and challenges discussed in the previous section also
apply to DSCs. Implementing DSCs is considered to have a multitude of
benefits
\cite{buyukozkanDigital,kimImpact,korpelaDSC,maullDistributed,neubertCollaboration,santosOrganizational}
such as improving SC speed and effectiveness, reducing costs
throughout the chain, decreasing environmental impact and providing
flexibility in designing SCs. B\"uy\"uk\"ozkan and G\"o\c{c}er
\cite{buyukozkanDigital} indicate thet all companies regardless of
size or domain will need to explore digitalizing their supply chain to
be able to compete in the future. Additionally, Ivanov and Dolgui
\cite{ivanovIntertwined} demonstrate how participants in SCs become
interconnected to each other, and as an effect of globalization, to
the participants of other SCs as well. According to Ivanov and Dolgui,
the resulting "\textit{Intertwined supply networks}" or ISNs must be
resilient to such events as the COVID-19 epidemic, else the effects
reverberating throughout the network will be critical to securing
goods and services globally. The authors consider utilizing
digitalization within these networks a promising approach to this
problem.

\paragraph{Challenges} The challenges associated with creating DSCs,
or converting to DSCs from SCs, are significant.
\cite{panInteroperability} and \cite{xuManaging} consider the primary
challenges to be involving interoperability of different SC
participants' information and IoT systems and also integration between
digital and non-digital systems. Consequently, the challenges
concerning IoT are prevalent for DSCs, too. See the previous section
for those. For a thorough listing of challenges, see the literature
review by B\"uy\"uk\"ozkan and G\"o\c{c}er \cite{buyukozkanDigital}.

\paragraph{Blockchain solutions} Korpela et al. \cite{korpelaDSC}
explore how blockchains could be used to drive B2B
(\textit{business-to-business}) system integration within DSCs. They
argue that a layer of interoperability between companies' systems
could be created by utilizing blockchain while providing transparency
to the current state of the SC. Specifically the blockchain ledger and
the ability to process smart contracts were found by the authors'
survey of SC managers to be valuable for DSCs. Dutta et al.
\cite{duttaReview} note that interoperability is challenging for
blockchains as well since many different blockchain systems exist and
there is not yet sufficient standardization in creating blockchains.
Yet more papers \cite{coleBlockchain,hackiusLogistics} consider
blockchain to improve contract processing, tracking the origin of raw
materials, preventing forgery of data and more. Especially the paper
by Cole et al. \cite{coleBlockchain} exudes different posed benefits
of blockchains in DSCs. Babich and Hilary note that though many
existing solutions could be used for solving problems in DSCs, they
may often be less economically feasible than using a blockchain
\cite{babichDistributed}.

Cole et al. do also notice that adoption of blockchain can have a lot
of friction since it would require large investments. Sternberg et al.
\cite{sternbergStruggle} expand on this by pointing out that the value
of blockchain for a single SC participant is \textit{"[.] at best
limited because the benefits are reaped when a critical mass of
stakeholders and value chain partners adopt the technology."}
Sternberg et al. also introduce the \textit{"Trust-investment
paradox"} which asserts that for investments into blockchain by SC
participants to occur, the participants must have trust in each other.
If participants already have trust in each other, blockchain is not
required for building trust. 

Blockchain also has its weaknesses once it is already \textit{in} use.
These are largely weaknesses relating to IoT such as the problem of
inefficiency and the GIGO problem discussed in the previous section.
Babich and Hilary \cite{babichDistributed} additionally mention the
lack of standardization and the \textit{black-box effect}. This refers
to the idea that even if transparency into the internal state of the
blockchain is achieved, the users of the blockchain still need to
trust the technology itself. And if trust on the technology is
limited, its adoption will suffer, leading to negative effects
on adoption throughout the supply chain \cite{sternbergStruggle}.


\paragraph{Alternative solutions}

Many business processes in SCM could be digitalized using an ERP
(\textit{enterprise resource planning}) system. ERP has many
functionalities similar to blockchain such as automatic smart contract
processing but it is centrally managed which allows for increased
efficiency. However, its functions are limited to the confines of a
single organization which has negatively impacted adoption in SCM
\cite{coleBlockchain}.

Solutions in IIoT are largely applicable to the domain of SCM as well
since IoT is an identified method for driving DSC conversion
\cite{buyukozkanDigital}. It follows that, equivalently to the IoT
section, distributed DBMS could be applied to many DSC use cases as
well to drive data integration. Using a centrally-managed database
would also not have the problem of lacking standardization or limited
trust towards the technology as DBMS are well standardized and
trusted.

Thus, if operating in a full-trust environment, such as a single
company's SC or the company's internal processes, an ERP system that
utilizes a distributed DBMS is suitable.


\paragraph{Conclusion}

Generally, solutions in the domain of SCM need not be concerned with
requiring public writability. As such, using permissionless
Blockchains should generally be avoided. Consequently, the conclusion
of using a permissioned blockchain like Hyperledger Fabric in IIoT
applies here, too. Furthermore, Hyperledger Fabric is an established
framework within blockchain, which provides the \textit{meta-trust}
required against the black-box effect mentioned by Babich and Hilary
\cite{babichDistributed}.

Once again, if possible, using a centralized system is recommended.
However, as argued in \cite{coleBlockchain}, this is often not the
case in SCM because multiple independent oragnizations need to access the
same data.


\begin{table}[!ht]
\begin{tabularx}{\textwidth}{lXXX}
Solution                                                        & Hyperledger Fabric & ERP system       \\[2ex] \hline
\begin{tabular}[c]{@{}l@{}}Trust \\ required\end{tabular}       & limited            & full             \\[3ex] \hline
\begin{tabular}[c]{@{}l@{}}Permission\\ required\end{tabular}   & yes                & yes              \\[3ex] \hline
\begin{tabular}[c]{@{}l@{}}Implementing\\ security\end{tabular} & cheap              & expensive        \\[3ex] \hline
\begin{tabular}[c]{@{}l@{}}Scaling:\\ Network size\end{tabular} & moderate           & low              \\[3ex] \hline
\begin{tabular}[c]{@{}l@{}}Scaling:\\ TPS\end{tabular}          & high               & optimal          \\[3ex] \hline
\begin{tabular}[c]{@{}l@{}}GIGO\\ problem\end{tabular}          & solvable           & inexistent       \\[3ex] \hline
\begin{tabular}[c]{@{}l@{}}Suggested\\ area\end{tabular}        & Multi-org SCM      & Private SCM      \\[3ex] \hline
\end{tabularx}
\caption{Results of assessing blockchain applicability in SCM}
\label{tab:scm}
\end{table}