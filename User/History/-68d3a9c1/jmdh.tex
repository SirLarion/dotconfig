\section{Applications of blockchains}

Given the strengths and weaknesses of blockchains, what can they be
used for in practice? I will spend this chapter reviewing blockchain
usage within different domains. Each domain is covered in its own
subsection, where I present problems that the domain has faced, and
how blockchain has been applied in attempt to fix those problems. If
possible, I will also go over the alternative. That is, how solutions
that do not involve blockchain have been similarly applied. 

It must be noted that empiric research of blockchain performance in
different domains (excepting, perhaps, the domain of finance) is quite
insufficient for drawing conclusions. Due to this, many of the
references made in this section will be to case studies done by the
institution applying blockchain, or another institution that has a
vested interest in the application (for example, the Hyperledger
Foundation).

When reviewing the applicability of a novel technology such as
blockchain, it must be kept in mind that the problems that are being
solved might also be solved with another, existing technology. This
has been a shortcoming of many research papers that propose the usage
of blockchain in different areas. To this end, I will relate given
examples of blockchain applications back to other applications that
\textit{do not} use blockchain, given such ones exist.

...

\subsection{Payment systems}

Using cryptocurrencies as means of payment is an attractive idea as
it, theoretically, combines the privacy of cash payments and the
convenience of current digital payments. However, this idea faces a
few severe problems, namely scalability and market volatility.

Sanka and Cheung \cite{sankaScalability} discuss the problem of
blockchain scalability extensively. They introduce the so called
"\textit{Blockchain Scalability Quadrilemma}" that defines a 4-way
tradeoff between scalability, trust, security and decentralization.
This quadrilemma poses that any improvement in one of the properties
must go along with a decline in one of the others. While scalability
is a prevalent problem for the technology itself, it is highlighted
when using blockchain for payment systems. For a payment system to
have any meaningful impact, it must be open to the public, requiring a
high degree of scalability. Additionally, being public means that the
chain must be permissionless and thus security is a major concern.
Already, assuming the existence of the quadrilemma, it is evident that
two of the four properties are locked in place. As such, either trust
or decentralization must yield to the other. This is shown later in
practice by the different types of payment systems that have been
built on blockchain.

Making currencies averse to volatility is a topic that has been
studied for decades, if not centuries. The more volatility a currency
experiences, the less trust can be placed on its purchasing power. So
called "stablecoins", as the name implies, are stores of monetary
value that are created solely with the goal of maintaining stability
with respect to a currency or currencies \cite{liptonTether}. The
origin of stablecoins lies in cryptocurrencies but, as discussed in
\cite{liptonTether}, their meaning has expanded to a larger context of
electric money. I will, however, only discuss stablecoins as an
application of blockchains.

All stablecoins have some mechanism for value stabilization. This way
stablecoin holders can place more trust in the value of their asset.
Mita et. al. \cite{mitaStablecoin} discuss the existing stablecoins
and their ways of attaining stability. The system for stabilizing a
coin's value can be based on providing other assets (US dollars, other
cryptocurrencies or gold, for example) as collateral. Stabilization
can also be algorithmic, or \textit{non-collaterized}, meaning that
the amount of stablecoins in circulation is automatically controlled
by the underlying blockchain. 

Mita et. al. argue that only non-collaterized stablecoins can have the
outcome that is desired as collaterization with fiat currencies or
commodities would result in centralization (decline of
decentralization in the quadrilemma of Sanka and Cheung
\cite{sankaScalability}). Collaterization with other cryptocurrencies
in turn would not actually stabilize the value of the currency. The
authors do also conclude that current non-collaterized stabilization
mechanisms cannot guarantee the stability of \textit{purchasing power}
which is desired when using them for payments. A paper by Eichengreen
\cite{eichengreenCommodity}, however, takes a firm stance against
non-collaterization. Eichengreen argues that non-collaterized
stablecoins cannot have resilience against price fluctuations as they
lead to snowball effects that, without collateral, easily collapse the
value of the coin entirely (decline of trust in the quadrilemma).

Moreover, the blockchain-based stablecoins that are currently traded
most are collaterized. At the time of writing, data from CoinMarketCap
\footnote{https://coinmarketcap.com} shows that the top 10 stablecoins
by trading volume include three stablecoins that are listed as
"algorithmic" (Decentralized USD, Terra Classic USD and sUSD). These
account for about 0.6\% of the trading volume of the top 10 while the
top three (USD Tether, USD Coin and Binance USD) account for roughly
98\% of the same volume.

For blockchain applications in the space of payment systems, I will
only discuss stablecoins. This is because, currently, they seem to be
the only cryptocurrencies that are created with solving the volatility
problem in mind. Cryptocurrencies that do not intend to solve the
volatility problem should not be considered for use in payment as
explained in \cite{yermackBitcoin}.

Tether \footnote{https://tether.to}, specifically USD Tether (USDT),
is by far the most traded stablecoin, accounting for 77\% of the
volume of the top 10 stablecoins in CoinMarketCap statistics. In fact
its trading volume is higher than that of Bitcoin (BTC), the second
most traded cryptocurrency, by about 74\%. It is collaterized by fiat
currencies corresponding to the traded "\textit{Tether token}". For
example, USDT are tokens backed by US dollars and EURT respectively by
Euros. I will refer to the USDT token from now on for simplicity.
These Tether tokens lie on multiple different cryptocurrency
blockchains to create price stability. By all accounts, it does so
fairly well. Since the issuance of USDT in 2015 its deviations from
the price of USD have remained under ±10\% Figure
\ref{fig:tether-deviations} shows deviations in more detail. A study
by Lyons and Viswanath-Natraj \cite{lyonsStable} 

...

More examples

...

\begin{figure}
  \includegraphics[width=\linewidth]{tether-deviations.png}
  \caption{USDT/USD deviations from peg and histogram of deviations.
  As shown in \cite{lyonsStable}.}
  \label{fig:tether-deviations}
\end{figure}

As discussed earlier, collaterized stablecoins like Tether face
criticism in the blockchain space for what can be argued to be moving
towards centralization. Tether themselves have stated that
\textit{"Tether does not purport to be a central bank, and it is false
to suggest that Tether is like a central bank for a number of reasons
[.]"}\footnote{https://tether.to/en/a-commentary-on-tether-chainalysis}.
Clearly, Tether is not a central bank. However, it and other
collaterized stablecoins behave very similarly to a \textit{normal}
bank. Users store value of a certain amount of fiat currency by giving
it to a company for safekeeping. They do this as they have a belief
that they are able to retrieve it later when they need it. As long as
these companies are audited to show they have the collateral they
claim to have, there is no problem with this. In fact, this may have
groundbreaking effects on competition in the banking sector, which is,
traditionally, very difficult to enter as a newcomer. In comparison,
the \textit{go-to-market} for stablecoins has drastically lower
requirements.


...

\subsection{Digital supply chain management}

In markets where companies increasingly outsource parts of their
production to different, often international, entities, managing the
resulting supply chains (SC) is instrumental to all entities taking
part. A high degree of planning and communication is required to
respond to future demand for products and maintaining
cost-effectiveness throughout the chain. As demonstrated by Ivanov et.
al. \cite{ivanovIntertwined}, the participants become interconnected
to each other, and as an effect of globalization, to the participants
of other SCs as well. Ivanov et. al. conclude that the resulting
"\textit{Intertwined supply networks}" must be resilient to such
events as the COVID-19 epidemic, else the effects reverberating
throughout the network will be critical to securing goods and
services globally. In practice, this resilience is built with interoperability.

As such, supply chain management (SCM) is an area which would
certainly benefit from digitalization. Korpela et. al.
\cite{korpelaDSC} discuss the potential benefits of digital supply
chains (DSCs). These include improving SC effectiveness, reducing
costs throughout the chain and providing flexibility in designing
them. Done properly, digitalization would also decrease human error as
a consequence of automation.  However, converting an SC to a DSC is
quite a difficult problem. A fully digitalized SC would require
interoperability between the internal systems of \textit{all} the
participants in the SC.

Korpela et. al. go on to explore potential benefits of using
blockchains to drive business-to-business (B2B) system integration
within DSCs. They argue that in DSCs, creating a blockchain layer
would allow for interoperability between businesses while providing
visibility to the current state of the SC.

A permissioned blockchain enables a DSC to operate using trusted
sensors whose data are added to the ledger for SC participants to see.
They can then be confident that the data has not been tampered with.
However, what if the \textit{environment} of the sensor is tampered
with? \cite{wustBlockchainNeed} gives an example of a blockchain-based
DSC for food. In one part of it, food is meant to be stored in a
certain temperature during delivery. Figure \ref{fig:scm-problem}
illustrates the posed problem. The trusted sensor that is connected to
the blockchain is placed in a cooled compartment of the truck, whereas
the food itself is not. This way, a dishonest participant of the SC
can cut costs by having reduced cooling for the food while the
blockchain indicates it was properly cooled. This problem can, of
course, be fixed by verifying through some third party that the
sensors are installed correctly. But if such a system of verification
is available, why not simply use a regular database management system
for reading and writing the data from the sensors?

\begin{figure}
  \includegraphics[width=\linewidth]{scm-problem.png}
  \caption{A potential problem with trusted sensors in an SC
  blockchain as posed by \cite{wustBlockchainNeed}. The left shows the
  intended use of the trusted sensor (illustrated as a snowflake) and
  the right
  shows the malicious use.}
  \label{fig:scm-problem}
\end{figure}

Despite the existence of such hypothetical problem scenarios, many
companies have found success improving their SCs by using a blockchain
application. The Hyperledger Foundation, which has grown quite popular
as a maintainer and provider of blockchain frameworks geared at
enterprises, has conducted several case studies (58 listed on their
website as of writing
\footnote{https://www.hyperledger.org/learn/case-studies}) on
blockchain applications using their suite of frameworks.

One such case study investigates how Hitachi used the Hyperledger
Fabric blockchain framework to improve their internal contract
processing procedure \cite{hyperledgerHitachi}. They started from a
completely paper-based system and converted a division to use a fully
electric signature system built on Hyperledger Fabric. As a result,
Hitachi reported going from 333 to over 400 monthly processed
contracts, a 20\% increase.

Another study looks at Walmart Canada working to reduce invoice
disputes between them and their carriers, the companies that are
contracted to deliver goods for them \cite{hyperledgerWalmart}.
Walmart Canada ordered a tailor-made blockchain application for
this purpose. The resulting system reportedly enables them to share
real-time data between them and all 70 carriers. The study concludes
that the amount of invoice disputes dropped from around 70\% of
invoices to 1.5\%. A decrease of 98\%.
 
In contrast, a case study very similar to that of Hitachi
\cite{docusignBurger} explains how Restaurant Services, Inc. (RSI),
the supply chain manager for the Burger King Corporation (BKC),
applied a set of non-blockchain technologies to the same problem of
converting to a paperless contract signing procedure. At the core of
the conversion was DocuSign's eSignature service. It provides users a
way to sign contracts digitally by creating a hash of the contents of
signed contracts via symmetric-key cryptography. DocuSign then stores
the hash inside a database so that any tampering of copies of the
contract is evident when comparing against it \cite{docusignHow}. A
team of two developers digitalized the entirety of BKC's contract
signing procedure.
