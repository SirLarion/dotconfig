% ---------------------------------------------------------------------
% -------------- PREAMBLE ---------------------------------------------
% ---------------------------------------------------------------------
\documentclass[12pt,a4paper,finnish,oneside]{article}
%\documentclass[12pt,a4paper,finnish,twoside]{article}
%\documentclass[12pt,a4paper,finnish,oneside,draft]{article} % luonnos, nopeampi

% Valitse 'input encoding':
%\usepackage[latin1]{inputenc} % merkistökoodaus, jos ISO-LATIN-1:tä.
\usepackage[utf8]{inputenc}   % merkistökoodaus, jos käytetään UTF8:a
% Valitse 'output/font encoding':
%\usepackage[T1]{fontenc}      % korjaa ääkkösten tavutusta, bittikarttana
\usepackage{ae,aecompl}       % ed. lis. vektorigrafiikkana bittikartan sijasta
% Kieli- ja tavutuspaketit:
\usepackage[finnish,swedish,english]{babel}
% Kurssin omat asetukset aaltosci_t.sty:
% \usepackage{aaltosci_t}
% Jos kirjoitat muulla kuin suomen kielellä valitse:
%\usepackage[finnish]{aaltosci_t}           
%\usepackage[swedish]{aaltosci_t}           
\usepackage[english]{aaltosci_t}           
% Muita paketteja:
\usepackage{alltt}
\usepackage{amsmath}   % matematiikkaa
\usepackage{calc}      % käytetään laskurien (counter) yhteydessä (tiedot.tex)
\usepackage{eurosym}   % eurosymboli: \euro{}
\usepackage{url}       % \url{...}
\usepackage{listings}  % koodilistausten lisääminen
\usepackage{algorithm} % algoritmien lisääminen kelluvina
\usepackage{algorithmic} % algoritmilistaus
\usepackage{hyphenat}  % tavutuksen viilaamiseen liittyvä (hyphenpenalty,...)
\usepackage{tabularx}

% Koko dokumentin kattavia asetuksia:

% Tavutettavia sanoja:
%\hyphenation{vää-rin me-ne-vi-en eri-kois-ten sa-no-jen tavu-raja-ehdo-tuk-set}
% Huomaa, että ylläoleva etsii tarkalleen kyseisiä merkkijonoja, eikä
% ymmärrä taivutuksia. Paikallisesti tekstin seassa voi myös ta\-vut\-taa.

% Rangaistaan tavutusta (ei toimi?! Onko hyphenat-paketti asennettu?)
\hyphenpenalty=10000   % rangaistaan tavutuksesta, 10000=ääretön
\tolerance=1000        % siedetään välejä riveillä
% titlesec-paketti auttaa, jos tämän mukana menee sekaisin

% Tekstiviitteiden ulkoasu.
% Pakettiin natbib.sty/aaltosci.bst liittyen katso esim. 
% http://merkel.zoneo.net/Latex/natbib.php
% jossa selitykset citep, citet, bibpunct, jne.
% Valitse alla olevista tai muokkaa:
% \bibpunct{(}{)}{;}{a}{,}{,}    % a = tekijä-vuosi (author-year)
\bibpunct{[}{]}{,}{n}{,}{,}    % n = numero [1],[2] (numerical style)

% Rivivälin muuttaminen:
\linespread{1.24}\selectfont               % riviväli 1.5
%\linespread{1.24}\selectfont               % riviväli 1, kun kommentoit pois

\setlength{\skip\footins}{2pc plus 5pt}
\setlength{\footnotesep}{1pc}
% ---------------------------------------------------------------------
% -------------- DOCUMENT ---------------------------------------------
% ---------------------------------------------------------------------

\begin{document}

% -------------- Tähän dokumenttiin liittyviä valintoja  --------------

%\raggedright         % Tasattu vain vasemmalta, ei tavutusta
\input{makroja}       % Haetaan joitakin makroja

% Kieli:
% Kielesi, jolla kandidaatintyön kirjoitat: finnish, swedish, english.
% Tästä tulee mm. tietyt otsikkonimet ja kuva- ja taulukkoteksteihin 
% (Kuva, Figur, Figure), (Taulukko, Tabell, Table) sekä oikea tavutus.
% \selectlanguage{finnish}
%\selectlanguage{swedish}
\selectlanguage{english}

% Sivunumeroinnin kanssa pieniä ristiriitaisuuksia.
% Toimitaan pääosin lähteen "Kirjoitusopas" luvun 5.2.2 mukaisesti.
% Sivut numeroidaan juoksevasti arabialaisin siten että 
% ensimmäiseltä nimiölehdeltä puuttuu numerointi.
\pagestyle{plain}
\pagenumbering{arabic}
% Muita tapoja: kandiohjeet: ei numerointia lainkaan ennen tekstiosaa
%\pagestyle{empty}
% Muita tapoja: kandiohjeet: roomalainen numerointi alussa ennen tekstiosaa
%\pagestyle{plain}
%\pagenumbering{roman}        % i,ii,iii, samalla alustaa laskurin ykköseksi

% ---------------------------------------------------------------------
% -------------- Luettelosivut alkavat --------------------------------
% ---------------------------------------------------------------------

% -------------- Nimiölehti ja sen tiedot -----------------------------
%
% Nimiölehti ja tiivistelmä kirjoitetaan seminaarin mukaan joko
% suomeksi tai ruotsiksi (ellei erityisesti kielenä ole englanti). 
% Tiivistelmän voi suomen/ruotsin lisäksi kirjoittaa halutessaan
% myös englanniksi. Eli tiivistelmiä tulee yksi tai kaksi kpl.
%
% "\MUUTTUJA"-kohdat luetaan aaltosci_t.sty:ä varten.

\author{Miska Tammenpää}

% Otsikko nimiölehdelle. Yleensä sama kuin seuraavana oleva \TITLE, 
% mutta jos nimiölehdellä tarvetta "kaksiosaiselle" kaksiriviselle
\title{The many faces of Blockchain: Strengths, limitations and applications}
% 2-osainen otsikko:
%\title{\LaTeX{}-pohja kandidaatintyölle \\[5mm] Pitkiä rivejä kokeilun vuoksi.}

% Otsikko tiivistelmään. Jos lisäksi engl. tiivistelmä, niin viimeisin:
\TITLE{The many faces of Blockchain: strengths, limitations and applications}
%\TITLE{\LaTeX{} för kandidatseminariet med jättelång rubrik som fortsätter och
% fortsätter ännu}
\ENTITLE{The many faces of Blockchain: strengths, limitations and applications}
% 2-osainen otsikko korvataan täällä esim. pisteellä:
%\TITLE{\LaTeX{}-pohja kandidaatintyölle. Pitkiä rivejä kokeilun vuoksi.}

% Ohjaajan laitos suomi/ruotsi ja tarvittaessa eng (tiivistelmän kieli/kielet)
% \DEPT{Poimi tähän ohjaajasi laitos, DEPT, main.tex}
% suomi:
\DEPT{Tietotekniikan laitos}               % T
%\DEPT{Tietojenkäsittelytieteen laitos}     % TKT
%\DEPT{Mediatekniikan laitos}               % ME
% ruotsi:
%\DEPT{Institutionen för datateknik}        % T
%\DEPT{Institutionen för datavetenskap}     % TKT
%\DEPT{Institutionen för mediateknik}       % ME
% englanti:
\ENDEPT{Department of Computer Science Engineering}     % T
%\ENDEPT{Department of Information and Computer Science} % TKT
%\ENDEPT{Department of Media Technology}                 % ME

% Vuosi ja päivämäärä, jolloin työ on jätetty tarkistettavaksi.
\YEAR{2022}
\DATE{dd. monthmonth 2022}
%\DATE{31. helmikuuta 2011}
%\DATE{Den 31 februari 2011}
\ENDATE{monthmonth dayday, 2022}

% Kurssin vastuuopettaja ja työsi ohjaaja(t)
\SUPERVISOR{Professori Eero Hyvönen}
\INSTRUCTOR{Tutkimusapulainen Carl Kugblenu}
%\INSTRUCTOR{Ohjaajantitteli Sinun Ohjaajasi, ToinenTitt Matti Meikäläinen}
% DI       // på svenska DI diplomingenjör
% TkL      // TkL teknologie licentiat
% TkT      // TkD teknologie doctor
% Dosentti Dos. // Doc. Docent
% Professori Prof. // Prof. Professor
% 
% Jos tiivistelmä englanniksi, niin:
\ENSUPERVISOR{Professor Eero Hyvönen}
\ENINSTRUCTOR{Doctoral Candidate Carl Kugblenu}
% M.Sc. (Tech)  // M.Sc. (Eng)
% Lic.Sc. (Tech)
% D.Sc. (Tech)   // FT filosofian tohtori, PhD Doctor of Philosophy
% Docent
% Professor

% Kirjoita tänne HOPS:ssa vahvistettu pääaineesi.
% Pääainekoodit TIK-opinto-oppaasta.

\PAAAINE{Computer Science and Engineering}
%Tietotekniikka
%Datateknik
%Computer Science and Engineering
\CODE{JOIN.kand}

%\PAAAINE{Ohjelmistotuotanto ja -liiketoiminta}
%\CODE{T3003}
%
%\PAAAINE{Tietoliikenneohjelmistot}
%\CODE{T3005}
%
%\PAAAINE{WWW-teknologiat} % vuodesta 2010
%\CODE{IL3012}
%
%\PAAAINE{Mediatekniikka} % vuoteen 2010, kts. seur.
%\CODE{T3004}
%
%\PAAAINE{Mediatekniikka} % vuodesta 2010, kts. edell.
%\CODE{IL3011}
%
%\PAAAINE{Tietojenkäsittelytiede} % vuodesta 2010
%\CODE{IL3010}
%
%\PAAAINE{Informaatiotekniikka} % vuoteen 2010
%\CODE{T3006}
%
%\PAAAINE{Tietojenkäsittelyteoria} % vuoteen 2010
%\CODE{T3002}
%
%\PAAAINE{Ohjelmistotekniikka}
%\CODE{T3001}

% Avainsanat tiivistelmään. Tarvittaessa myös englanniksi:

\KEYWORDS{avain, sana}
\ENKEYWORDS{key, word}

% Tiivistelmään tulee opinnäytteen sivumäärä.
% Kirjoita lopulliset sivumäärät käsin tai kokeile koodia. 
%
% Ohje 29.8.2011 kirjaston henkilökunnalta:
%   - yhteissivumäärä nimiölehdeltä ihan loppuun
%   - "kaikkien yksinkertaisin ja yksiselitteisin tapa"
%
% VANHA // Ohje 14.11.2006, luku 4.2.5:
% VANHA // - sivumäärä = tekstiosan (alkaen johdantoluvusta) ja 
% VANHA //  lähdeluettelon sivumäärä, esim. "20"
% VANHA // - jos liitteet, niin edellisen lisäksi liitteiden sivumäärä,
% VANHA //  tyyli "20 + 5", jossa 5 sivua liitteitä 
% VANHA // - HUOM! Tässä oletuksena sivunumerointi alkaa nimiölehdestä 
% VANHA //  sivunumerolla 1. %   Toisin sanoen, viimeisen lähdeluettelosivun 
% VANHA //  sivunumero EI ole sivujen määrä vaan se pitää laskea tähän käsin

\PAGES{x}
%\PAGES{23}  % kaikki sivut laskettuna nimiölehdestä lähdeluettelon tai 
             % mahdollisten liitteiden loppuun. Tässä 23 sivua

%\thispagestyle{empty}  % nimiölehdellä ei ole sivunumerointia; tyylin mukaan ei tehdäkään?!

\maketitle             % tehdään nimiölehti

% -------------- Tiivistelmä / abstract -------------------------------
% Lisää abstrakti kandikielellä (ja halutessasi lisäksi englanniksi).

% Edelleen sivunumerointiin. Eräs ohje käskee aloittaa sivunumeroiden
% laskemisen nimiösivulta kuitenkin niin, että sille ei numeroa merkitä
% (Kauranen, luku 5.2.2). Näin ollen ensimmäisen tiivistelmän sivunumero
% on 2. \maketitle komento jotenkin kadottaa sivunumeronsa.
\setcounter{page}{2}    % sivunumeroksi tulee 2

\input{luku_abstraktit}
\newpage                       % pakota sivunvaihto

% -------------- Sisällysluettelo / TOC -------------------------------

\tableofcontents

\label{pages:prelude}
\clearpage                     % kappale loppuu, loput kelluvat tänne, sivunv.
%\newpage

% -------------- Symboli- ja lyhenneluettelo -------------------------
% Lyhenteet, termit ja symbolit.
% Suositus: Käytä vasta kun paljon symboleja tai lyhenteitä.
%
% \input{luku_lyhenteet} 
% \clearpage                     % luku loppuu, loput kelluvat tänne
% \newpage

% -------------- Kuvat ja taulukot ------------------------------------
% Kirjoissa (väitöskirja) on usein tässä kuvien ja taulukoiden listaus.
% Suositus: Ei kandityöhön.

% -------------- Alkusanat --------------------------------------------
% Suositus: ÄLÄ käytä kandidaatintyössä. Jos käytät, niin omalle 
% sivulleen käyttäen tarvittaessa \newpage
%
%\input{luku_alkusanat}
%\clearpage                     % luku loppuu, loput kelluvat tänne
%\newpage                       % pakota sivunvaihto
%
%SH: Alkusanoissa voi kiittää tahoja, jotka ovat merkittävästi edistäneet
% työn valmistumista. Tällaisia voivat olla esimerkiksi yritys, jonka
% tietokantoja, kontakteja tai välineistöä olet saanut käyttöösi,
% haastatellut henkilöt, ohjaajasi tai muut opettajat ja myös
% henkilökohtaiset kontaktisi, joiden tuki on ollut korvaamatonta työn
% kirjoitusvaiheessa. Alkusanat jätetään tyypillisesti pois
% kandidaatintyöstä, joka on laajuudeltaan vielä niin suppea, ettei
% kiiteltäviä tahoja luontevasti ole.

% ---------------------------------------------------------------------
% -------------- Tekstiosa alkaa --------------------------------------
% ---------------------------------------------------------------------

% Muutetaan tarvittaessa ala- ja ylätunnisteet
%\pagestyle{headings}          % headeriin lisätietoja
%\pagestyle{fancyheadings}     % headeriin lisätietoja
%\pagestyle{plain}             % ei header, footer: sivunumero

% Sivunumerointi, jos käytetty 'roman' aiemmin
% \pagenumbering{arabic}        % 1,2,3, samalla alustaa laskurin ykköseksi
% \thispagestyle{empty}         % pyydetty ensimmäinen tekstisivu tyhjäksi

\label{pages:text}

% CONTENT
\input{introduction}
\clearpage

\input{background}
\clearpage

\input{strengths&limitations}
\clearpage

\input{applications}
\clearpage

% \input{future}
% \clearpage

\input{conclusion}
\clearpage

%\newpage                       % ellei ylempi tehoa, pakota lähdeluettelo 
                               % alkamaan uudelta sivulta

% -------------- Lähdeluettelo / reference list -----------------------
%
% Lähdeluettelo alkaa aina omalta sivultaan; pakota lähteet alkamaan
% joko \clearpage tai \newpage
%
%
% Muista, että saat kirjallisuusluettelon vasta
%  kun olet kääntänyt ja kaulinnut "latex, bibtex, latex, latex"
%  (ellet käytä Makefilea ja "make")

% Viitetyylitiedosto aaltosci_t.bst; muokattu HY:n tktl-tyylistä.
\bibliographystyle{aaltosci_t}
% Katso myös tämän tiedoston yläosan "preamble" ja siellä \bibpunct.

% Muutetaan otsikko "Kirjallisuutta" -> "Lähteet"
\renewcommand{\refname}{\REFERENCES}  % article-tyyppisen
%\renewcommand{\bibname}{Lähteet}  % jos olisi book, report-tyyppinen

% Lisätään sisällysluetteloon
\addcontentsline{toc}{section}{\refname}  % article
%\addcontentsline{toc}{chapter}{\bibname}  % book, report

% Määritä kaikki bib-tiedostot
\bibliography{lahteet}
%\bibliography{thesis_sources,ietf_sources}

\label{pages:refs}
\clearpage         % erotetaan mahd. liitteet alkamaan uudelta sivulta

% -------------- Liitteet / Appendices --------------------------------
%
% Liitteitä ei yleensä tarvita. Kommentoi tällöin seuraavat
% rivit.

% Tiivistelmässä joskus matemaattisen kaavan tarkempi johtaminen, 
% haastattelurunko, kyselypohja, ylimääräisiä kuvia, lyhyitä 
% ohjelmakoodeja tai datatiedostoja.

% \appendix
% \input{luku_liitteet}


% \label{pages:appendices}

% ---------------------------------------------------------------------

\end{document}
